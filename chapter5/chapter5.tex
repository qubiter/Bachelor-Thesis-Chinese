\chapter{附录}\label{chapter_appendix}
\graphicspath{{chapter5/figure/}}

鉴于部分代码过于冗长,笔者并不附出;笔者已将其托管到自己的Github上:(https://github.com/qubiter),有兴趣可以前去访问。

\section{第三章算法测试:MNIST数字识别MATLAB代码}
\begin{spacing}{1}
\begin{lstlisting}[language=Matlab][deletekeywords={input, beta, gamma}]
%% DeSTIN Algorithm
%% STEP I: DeSTIN Network Training
%% Data Input
% the MNIST data is included in the MNIST.mat, where the X is the training
% image, y is the corresponding label.
load('MNIST.mat');

% Select Picture for Training
T = 5000; % Training Sample Number
sel = randperm(size(X, 1));
sel = sel(1:T);

% Display data
% Used the displayData.m function
displayData(X(sel,1:400))

fprintf('Program paused. Press enter to continue.\n');
%pause;

%% Initializing

% Online Clustering Learning Rate
alpha = 0.05; beta = 0.05;

% Starvation Rate
gamma = 0.992;

%{
Parameters for one neuron: 
1. Centroid attributes: mu, sigma2:   I-by-J matrix
2. Belief State:   J-by-1 vector
3. Input data:     I-by-1 vector
4. Starvation Trace:   J-by-1 vector
%}

% We adopt 'struct' structure for each neuron
Dim_Neu = [125, 125, 41; 25, 25, 25];
for k = 1:3
    for i = 1:2^(k-1)
        for j = 1:2^(k-1)
            n = (4^(k-1)-1)/3+(2^(k-1))*(i-1)+j;
            Neuron(n).input = zeros(Dim_Neu(1,k),1);
            Neuron(n).belief = zeros(Dim_Neu(2,k),1);
            Neuron(n).mu = 0.4 + 0.2 * rand(Dim_Neu(1,k),Dim_Neu(2,k));
            Neuron(n).sigma2 = ones(Dim_Neu(1,k),Dim_Neu(2,k));
            Neuron(n).starve = ones(Dim_Neu(2,k),1);
        end
    end
end

%% DeSTIN Training Process
% turn on the Training switch
training = 1;

% set a 'scanning sequence' for the MNIST data
scan = [0,0;0,1;0,2;0,3;0,4;1,4;1,3;1,2;1,1;1,0;2,0;2,1;...
2,2;2,3;2,4;3,4;3,3;3,2;3,1;3,0;4,0;4,1;4,2;4,3;4,4];
S = size(scan,1);
  
for t = 1:T % training sample sequence
    sample = reshape(X(sel(t),:), 20, 20);
    % scanning the sample with the specific sequence
    for s = 1:S  %scanning sequence
        sx = scan(s,1); sy = scan(s,2);
        for k = 3:-1:1  %for different layer
            for i = 1:2^(k-1)  % x label
                for j = 1:2^(k-1) % y label
                    n = (4^(k-1)-1)/3+(2^(k-1))*(i-1)+j; % calculating the neuron order number
                    if s == 1
                        Neuron(n).belief = 0.1 * ones(Dim_Neu(2,k),1); % reset the belief state at the beginning of each scanning; the belief state will be re-scaled later
                    end
                    % data transition
                    if k == 3  % for the bottom layer
                        A = reshape(sample(4*i-3+sx:4*i+sx,4*j-3+sy:4*j+sy),16,1);
                        % scaling the belief according to samle's mean value
                        Neuron(n).belief = Neuron(n).belief * ( mean(sample(:))./mean(Neuron(n).belief(:)) );
                        % input signal = spatial input + feedback input
                        Neuron(n).input = [A ; Neuron(n).belief];
                    else   % for the upper layer
                        % the spatial input from lower layer
                        B = [Neuron(((4^k-1)/3+2^k*((2*i-1-1)))+(2*j-1)).belief;...
                            Neuron(((4^k-1)/3+2^k*((2*i-1-1)))+(2*j)).belief;...
                            Neuron(((4^k-1)/3+2^k*((2*i-1)))+(2*j-1)).belief;...
                            Neuron(((4^k-1)/3+2^k*((2*i-1)))+(2*j)).belief];
                        %scaling
                        Neuron(n).belief = Neuron(n).belief * ( mean(sample(:))./mean(Neuron(n).belief(:)) );
                        Neuron(n).input = [B ; Neuron(n).belief];
                    end
                    % DeSTIN learning process
                    % Used the OnlineClustering function
                    [Neuron(n).belief,Neuron(n).mu,Neuron(n).sigma2,Neuron(n).starve] = OnlineClustering(...
                        Neuron(n).input,Neuron(n).mu,Neuron(n).sigma2,Neuron(n).starve, alpha,beta,gamma,training);
                end
            end
        end
    end
end

fprintf('Training completed. Press enter to continue.\n');
%pause;

%% Step II: 2-layer Neural Network Training
%% Sample Rechoose
T = 5000; % Neural Network Training sample number
sel = randperm(size(X, 1));
sel = sel(1:T);

% Display data
displayData(X(sel,1:400))

fprintf('Program paused. Press enter to continue.\n');
%pause;

%% DeSTIN usage: Acquire the output belief
% Turn off the training switch
training = 0;

% DeSTIN Learning Process
% belief state output to be collected
Belief = zeros(T, 21, 100);
for t = 1:T
    sample = reshape(X(sel(t),:), 20, 20);
    % scanning the sample with the specific sequence
    for s = 1:S
        sx = scan(s,1); sy = scan(s,2);
        for k = 3:-1:1
            for i = 1:2^(k-1)
                for j = 1:2^(k-1)
                    n = (4^(k-1)-1)/3+(2^(k-1))*(i-1)+j;
                    if s == 1
                        Neuron(n).belief = 0.1 * ones(Dim_Neu(2,k),1);
                    end
                    % data transition
                    if k == 3
                        A = reshape(sample(4*i-3+sx:4*i+sx,4*j-3+sy:4*j+sy),16,1);
                        % scaling
                        Neuron(n).belief = Neuron(n).belief * ( mean(sample(:))./mean(Neuron(n).belief(:)) );
                        Neuron(n).input = [A ; Neuron(n).belief];
                    else
                        B = [Neuron(((4^k-1)/3+2^k*((2*i-1-1)))+(2*j-1)).belief;...
                            Neuron(((4^k-1)/3+2^k*((2*i-1-1)))+(2*j)).belief;...
                            Neuron(((4^k-1)/3+2^k*((2*i-1)))+(2*j-1)).belief;...
                            Neuron(((4^k-1)/3+2^k*((2*i-1)))+(2*j)).belief];
                        %scaling
                        Neuron(n).belief = Neuron(n).belief * ( mean(sample(:))./mean(Neuron(n).belief(:)) );
                        Neuron(n).input = [B ; Neuron(n).belief];
                    end
                    % DeSTIN learning process
                    [Neuron(n).belief,Neuron(n).mu,Neuron(n).sigma2,Neuron(n).starve] = OnlineClustering(...
                        Neuron(n).input,Neuron(n).mu,Neuron(n).sigma2,Neuron(n).starve, alpha,beta,gamma,training);
                    % belief state storage
                    if mod(s,6) == 0
                       % Concatenate the 6,12,18,24th Belief state for storage
                       Belief(t,n,25*fix(s/6)-24:25*fix(s/6)) = (Neuron(n).belief)';
                    end
                end
            end
        end
 %{
        % **OUTPUT BELIEF STATE STORAGE
        % Collect the Belief output of specific position
        if mod(s,6)==0
            % Concatenate the 6,12,18,24th Belief state for storage
            Belief(t,25*fix(s/6)-24:25*fix(s/6)) = (Neuron(1).belief)'; 
        end
%}
    end
end

Belief = reshape(Belief,T,2100);
fprintf('Learning completed. Press enter to continue.\n');
%pause;

%% Initialize the Neural Network Parameters
input_layer_size  = 2100;   % Input Images of Neuron output
hidden_layer_size = 169;   % 20 hidden units
num_labels = 10;          % 10 labels, from 1 to 10  (note that we have mapped "0" to label 10)

% Weight Parameter for the 2-layer neural network
initial_Theta1 = randInitializeWeights(input_layer_size, hidden_layer_size);
initial_Theta2 = randInitializeWeights(hidden_layer_size, num_labels);
% Unroll parameters 
initial_nn_params = [initial_Theta1(:) ; initial_Theta2(:)];

% Regularization Factor
lambda = 1;

%% Neural Network Training
% Iteration options
options = optimset('MaxIter', 200);

% Create "short hand" for the cost function to be minimized
costFunction = @(p) nnCostFunction(p, ...
                                   input_layer_size, ...
                                   hidden_layer_size, ...
                                   num_labels, Belief, y(sel), lambda);

% Convex Optimization
[nn_params, cost] = fmincg(costFunction, initial_nn_params, options);

% Obtain Theta1 and Theta2 back from nn_params
Theta1 = reshape(nn_params(1:hidden_layer_size * (input_layer_size + 1)), ...
                 hidden_layer_size, (input_layer_size + 1));

Theta2 = reshape(nn_params((1 + (hidden_layer_size * (input_layer_size + 1))):end), ...
                 num_labels, (hidden_layer_size + 1));

fprintf('Program paused. Press enter to continue.\n');
%pause;

% Visualize the weight
%fprintf('\nVisualizing Neural Network... \n')

%displayData(Theta1(:, 2:end));

fprintf('\nProgram paused. Press enter to continue.\n');
%pause;

%% STEP III : Predictment Judge
%% Select Picture for Prediction
T = 1000; % Prediction Sample Number
sel = randperm(size(X, 1));
sel = sel(1:T);

%% Belief Calculation
% DeSTIN Learning Process
% belief state output to be collected
% exactly as the former DeSTIN part
training = 0;
Belief = zeros(T,21,100);
for t = 1:T
    sample = reshape(X(sel(t),:), 20, 20);
    % scanning the sample with the specific sequence
    for s = 1:S
        sx = scan(s,1); sy = scan(s,2);
        for k = 3:-1:1
            for i = 1:2^(k-1)
                for j = 1:2^(k-1)
                    n = (4^(k-1)-1)/3+(2^(k-1))*(i-1)+j;
                    if s == 1
                        Neuron(n).belief = 0.1 * ones(Dim_Neu(2,k),1);
                    end
                    % data transition
                    if k == 3
                        A = reshape(sample(4*i-3+sx:4*i+sx,4*j-3+sy:4*j+sy),16,1);
                        % scaling
                        Neuron(n).belief = Neuron(n).belief * ( mean(sample(:))./mean(Neuron(n).belief(:)) );
                        Neuron(n).input = [A ; Neuron(n).belief];
                    else
                       B = [Neuron(((4^k-1)/3+2^k*((2*i-1-1)))+(2*j-1)).belief;...
                            Neuron(((4^k-1)/3+2^k*((2*i-1-1)))+(2*j)).belief;...
                            Neuron(((4^k-1)/3+2^k*((2*i-1)))+(2*j-1)).belief;...
                            Neuron(((4^k-1)/3+2^k*((2*i-1)))+(2*j)).belief];
                        %scaling
                        Neuron(n).belief = Neuron(n).belief * ( mean(sample(:))./mean(Neuron(n).belief(:)) );
                        Neuron(n).input = [B ; Neuron(n).belief];
                    end
                    % DeSTIN learning process
                    [Neuron(n).belief,Neuron(n).mu,Neuron(n).sigma2,Neuron(n).starve] = OnlineClustering(...
                        Neuron(n).input,Neuron(n).mu,Neuron(n).sigma2,Neuron(n).starve, alpha,beta,gamma,training);
                    % belief state storage
                    if mod(s,6) == 0
                       % Concatenate the 6,12,18,24th Belief state for storage
                       Belief(t,n,25*fix(s/6)-24:25*fix(s/6)) = (Neuron(n).belief)';
                    end
                end
            end
        end
        %{
        % Collect the Belief output of specific position
        if mod(s,6)==0
            Belief(t,25*fix(s/6)-24:25*fix(s/6)) = (Neuron(1).belief)'; 
        end
        %}
    end
end
Belief = reshape(Belief,T,2100);
% Used perdict.m script
pred = predict(Theta1, Theta2, Belief);
fprintf('\nTraining Set Accuracy: %f\n', mean(double(pred == y(sel))) * 100);
toc;


\end{lstlisting}
\end{spacing}


\section{第四章硬件设计:部分Verilog代码}
\subsection{NeuronUnit:框架代码}
\begin{spacing}{1}
\begin{lstlisting}[language=Verilog]
`timescale 1ns / 100ps

module neuronunit(dist, norm_dist, divider_flag, in, mu_ini, sigma2_ini, en_input, en_renew, en_divide, renew_flag, ini_para, clk, rst);
  output signed [15:0] dist;
  output signed [15:0] norm_dist;
  output divider_flag; //divider_flag
  input signed [15:0] in;
  input signed [15:0] mu_ini;
  input signed [15:0] sigma2_ini;
  input en_input; //sample input control
  input en_renew; //renew process indicator
  input en_divide; //divider work signal
  input renew_flag; //whether new value or old one
  input ini_para; //for mu sigma2 initialization
  input clk;
  input rst;
  
  reg signed [15:0] in_reg;
  reg signed [15:0] mu;
  reg signed [15:0] sigma2;
  reg signed [15:0] diff_reg;
  reg signed [15:0] diff2_reg;
  //reg signed [15:0] dist;
  //reg signed [15:0] norm_dist;
  //reg divider_flag; //divider flag
  
  wire signed [15:0] diff;
  wire signed [15:0] diff2;
  wire signed [15:0] mu0;
  wire signed [15:0] sigma20;
  wire signed [15:0] mu_new;
  wire signed [15:0] sigma2_new;
  ///wire signed  [15:0] diff2_diff; //diff2_reg - sigma2
  wire divider_flag;
  
  //mu,sigma2 data input control
  assign mu0 = ini_para? mu_ini : mu_new;
  assign sigma20 = ini_para? sigma2_ini : sigma2_new;
  
  //pure logical calculation
  assign diff = in_reg - mu;
  assign dist = diff2;
  //pure logical multiplier
  fixed_width_multiply_16bits multiplier(.a(diff), .b(diff), .c(diff2));
  
  //renew rule execution
  assign mu_new = renew_flag? (mu + (diff_reg >>> 3 )) : mu;
  ///assign diff2_diff = diff2_reg - sigma2;
  assign sigma2_new = renew_flag? (sigma2 + (diff2_reg - sigma2) / 8 ) : sigma2;
  
  //examplify the divider
  cordic_div_smaller divider(.cordic_div_flag(divider_flag),.quotient(norm_dist),.dividend(diff2_reg),.division(sigma2),.clk(clk),.rst(rst),.cordic_div_en(en_divide));
  
  //register transition description
  always @(posedge clk or posedge rst)
  begin
    if (rst) begin in_reg = 16'd0; mu = 16'd0; sigma2 = 16'd0; diff_reg = 16'd0; diff2_reg = 16'd0; /*divider_flag = 0; norm_dist = 16'd0;*/ end
    else
      begin
        //diff,diff2 to register
        diff_reg <= diff;
        diff2_reg <= diff2;
        
        //input register renew
        if (en_input) in_reg <= in;
        
        //mu sigma2 register renew
        if (en_renew || ini_para) begin mu <= mu0; sigma2 <= sigma20; end
      end
  end
endmodule
\end{lstlisting}
\end{spacing}

\subsection{NeuronUnit:测试代码}

\begin{spacing}{1}
\begin{lstlisting}[language=Verilog]
`timescale 1ns / 100ps

module neuronunit_test();
  
  
  wire signed [15:0] dist;
  wire signed [15:0] norm_dist;
  wire divider_flag;
  
  reg clk;
  reg rst;
  reg en_input;
  reg en_renew;
  reg en_divide;
  reg renew_flag;
  reg ini_para;
  
  reg signed [15:0] in;
  reg signed [15:0] mu_ini;
  reg signed [15:0] sigma2_ini;
  
  
  neuronunit neuronunit_exam(.dist(dist), .norm_dist(norm_dist), .divider_flag(divider_flag), .in(in), .mu_ini(mu_ini), .sigma2_ini(sigma2_ini), 
  .en_input(en_input), .en_renew(en_renew), .en_divide(en_divide), .renew_flag(renew_flag), .ini_para(ini_para), .clk(clk), .rst(rst));
  
  initial
  begin
    // initialzation
    clk = 0;
    rst = 0;
    en_input = 0;
    en_renew = 0;
    en_divide = 0;
    renew_flag = 0;
    ini_para = 0;
    in = 0;
    mu_ini = 0;
    sigma2_ini = 0;
    
    // reset
    #10
    rst = 1;
    #10
    rst = 0;
    
    // initialize mu, sigma2: 0.5, 1
    #10
    ini_para = 1;
    #10
    mu_ini = 8192;
    sigma2_ini = 16384;
    #10
    ini_para = 0;
    mu_ini = 0;
    sigma2_ini = 0;
    
    // 1st input:0
    #10
    in = 0;
    #10
    en_input = 1;
    #10
    en_input = 0;
    
    // 1st renew part
    #10
    renew_flag = 1;
    #10
    en_renew = 1;
    #10
    en_renew = 0;
    
    // 1st divide part
    #10
    en_divide = 1;
    #10
    en_divide = 0;
    #140
    
    
    // 2nd input:0
    #10
    in = 0;
    #10
    en_input = 1;
    #10
    en_input = 0;
    
    // 2nd renew part
    #10
    renew_flag = 0;
    #10
    en_renew = 1;
    #10
    en_renew = 0;
    
    // 2nd divide part at least #140
    #10
    en_divide = 1;
    #10
    en_divide = 0;
    #140
    
    // 3rd input:0.2
    #10
    in = 3277;
    #10
    en_input = 1;
    #10
    en_input = 0;
    
    // 3rd renew part
    #10
    renew_flag = 1;
    #10
    en_renew = 1;
    #10
    en_renew = 0;
    
    // 3rd divide part at least #140
    #10
    en_divide = 1;
    #10
    en_divide = 0;
    #140

    #100
    $stop;
  end
  
  //clock
  initial
  begin
  forever
    #5 clk = ~clk;
  end
endmodule

\end{lstlisting}
\end{spacing}


\subsection{Cordic除法器:框架代码}
\begin{spacing}{1}
\begin{lstlisting}[language=Verilog]
`timescale 1ns / 100ps
module cordic_div_smaller(cordic_div_flag, quotient, dividend, division, clk, rst, cordic_div_en);
  parameter M = 1 ; // dividend shift bits (<<) 
  parameter N = 13; // division shift bits (>>) M + N = FractionLength
  parameter COUNT = 14; //Maximum count number
  
  output reg signed [15:0] quotient;
  output reg cordic_div_flag;
  
  input signed [15:0] dividend;
  input signed [15:0] division;
  input clk;
  input rst;
  input cordic_div_en;
  
  reg signed [43:0] y; //if result collapsed,try to alloc more bits.
  wire signed [43:0] y_buf;
  wire signed [27:0] y_mv;
  wire signed [27:0] y_mv2;
  
  reg [27:0] x;
  wire signed [27:0] x_mv;
  wire signed [27:0] x_mv2;
  
  reg signed [15:0] z;
  
  integer count;
  integer state;
  
  //extend the dividend's bit number
  assign y_mv = {dividend, 12'h0};
  assign y_mv2 = y_mv <<< M;  // redifine y_mv to get a right representation of z
  assign x_mv = {division, 12'h0};
  assign x_mv2 = x_mv >>> N;  // redifine x_mv to get a right representation of z
  assign y_buf = (y_mv2[27])?{16'hffff, y_mv2}:{16'h0000, y_mv2}; 
  
  always@(posedge clk or negedge rst)
  begin
    if(rst)begin
      state <= 0;
      count <= COUNT;
      x <= 0;
      y <= 0;
      z <= 0;
    end
    else begin
      case(state)
        0: begin
          cordic_div_flag <= 0;
          if(cordic_div_en) begin
            x <= x_mv2;
            y <= y_buf;
            z <= 0;
            state <= 1;
          end
        end
        1: begin
          if(count >= 0)begin
            count <= count - 1;
            if(y==0)begin
              y <= y;
              z <= z;
            end
            else if(y<0)begin
              z <= z - (1<<count);
              y <= y + (x<<count);
            end
            else begin
              z <= z + (1<<count);
              y <= y - (x<<count);
            end 
          end
          else begin
            count <= COUNT;
            quotient <= z;
            cordic_div_flag <= 1;
            state <= 0;
          end
        end
      endcase
    end
  end
  
endmodule

\end{lstlisting}
\end{spacing}


\subsection{Cordic除法器:测试代码}
\begin{spacing}{1}
\begin{lstlisting}[language=Verilog]
`timescale 1ns / 100ps

module cordic_div_test_small();
  
  
  wire signed [15:0] quotient;
  wire cordic_div_flag;
  
  reg signed [15:0] dividend;
  reg signed [15:0] division;
  reg clk;
  reg rst;
  reg cordic_div_en;
  cordic_div_smaller div_exam(.cordic_div_flag(cordic_div_flag), .quotient(quotient), .dividend(dividend), .division(division), .clk(clk), .rst(rst), .cordic_div_en(cordic_div_en));
  
  initial
  begin
    clk = 0;
    rst = 0;
    cordic_div_en = 0;
    dividend = 0;
    division = 0;
    #10
    rst = 1;
    #10
    rst = 0;
    // 1st: 0.8/1.5  8738
    #10
    dividend = 13107;
    division = 24576;
    #10
    cordic_div_en = 1;
    #10
    cordic_div_en = 0;
    // 2nd: 0.03/0.8  629
    #180
    dividend = 492;
    division = 13107;
    #10
    cordic_div_en = 1;
    #10
    cordic_div_en = 0;
    // 3rd: 0.001/0.8  21
    #180
    dividend = 16;
    division = 13107;
    #10
    cordic_div_en = 1;
    #10
    cordic_div_en = 0;
    // 4th: 0.00001/0.8  0
    #180
    dividend = 0;
    division = 13107;
    #10
    cordic_div_en = 1;
    #10
    cordic_div_en = 0;
    // 5th: -0.01/0.8  -209
    #180
    dividend = -164;
    division = 13107;
    #10
    cordic_div_en = 1;
    #10
    cordic_div_en = 0;
    // 6th: 0.024/0.049  7809
    #180
    dividend = 393;
    division = 817;
    #10
    cordic_div_en = 1;
    #10
    cordic_div_en = 0;
    
    #200
    $stop;
  end
  initial
  begin
  forever
    #5 clk = ~clk;
  end
endmodule
\end{lstlisting}
\end{spacing}